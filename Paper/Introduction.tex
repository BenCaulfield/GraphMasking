\section{Introduction}

\indent Social networking websites provide means to create and maintain meaningful connections between people. Because these websites hold all the data on connections between users, they too hold the tools to suggest future connections; however, this information must be used carefully. Networking websites are trusted with users' private information, and they must take care to not reveal any such information without the users' consent. Any new recommendation service must also reflect this responsibility.

\indent One such recommendation service is the public display of a social network's  \emph{k-neighborhood graph}, which shows a connection between two users if there is a path of $k$ or less between these users in the original network. Ideally,  users could look at this graph and make new connections based on their k-neighbors. However, it is possible than an adversary could use this information to determine a connection between users that should be kept private. 

\indent This paper presents a method to find the minimum $k$ value so that a graph cannot be determined from only its $k$-neighborhood. In the first section, we present a greedy algorithm which continuously adds edges to a graph until no edges can be added without invalidating the given $k$-neighborhood. The next section presents an algorithm to randomize a graph while maintaining its $k$-neighborhood. In section $NAME SECTION$, we give a heuristic to better randomize a given graph. The final section uses this algorithm to find the minimum randomizing $k$ value and uses the greedy algorithm to show that edges are properly disguised. 

\subsection {Relevant Work}

\indent There have been some studies on disguising networks of users so that only certain information is revealed. In 2002, Sweeney proposed the $k$-$anonymity$ model, which states that a presentation of data is $k$-$anonymous$ if any given user is indistinguishable from at least $k-1$ other individuals in the presentation \cite{Sweeney02}. In order to guarantee $k$-anonymity, Chester and Srivastava gave a method to alter a given network so that the resulting graph is $k$-anonymous \cite{Chester11}. Later, Chester et. al. study the complexity of this process of $k$-anonymization and define $t$-$closeness$, which measures how well an adversary could determine private information from an anonymized graph \cite{Chester13}.